\documentclass{article}

% ready for submission
\usepackage{arxiv}

% to compile a camera-ready version, add the [final] option, e.g.:
% \usepackage[final]{neurips_2018}

\usepackage[utf8]{inputenc} % allow utf-8 input
\usepackage[T1]{fontenc}    % use 8-bit T1 fonts
\usepackage{hyperref}       % hyperlinks
\usepackage{url}            % simple URL typesetting
\usepackage{booktabs}       % professional-quality tables
\usepackage{amsfonts}       % blackboard math symbols
\usepackage{nicefrac}       % compact symbols for 1/2, etc.
\usepackage{microtype}      % microtypography
\usepackage{graphicx}

\title{Why are financial markets particularly vulnerable to the madness of crowds?}

\date{January 16, 2021}


% The \author macro works with any number of authors. There are two commands
% used to separate the names and addresses of multiple authors: \And and \AND.
%
% Using \And between authors leaves it to LaTeX to determine where to break the
% lines. Using \AND forces a line break at that point. So, if LaTeX puts 3 of 4
% authors names on the first line, and the last on the second line, try using
% \AND instead of \And before the third author name.

\author{%
  Aidan Rocke\\
  \texttt{aidanrocke@gmail.com} \\
  % examples of more authors
  % \And
  % Coauthor \\
  % Affiliation \\
  % Address \\
  % \texttt{email} \\
  % \AND
  % Coauthor \\
  % Affiliation \\
  % Address \\
  % \texttt{email} \\
  % \And
  % Coauthor \\
  % Affiliation \\
  % Address \\
  % \texttt{email} \\
  % \And
  % Coauthor \\
  % Affiliation \\
  % Address \\
  % \texttt{email} \\
}

\begin{document}
% \nipsfinalcopy is no longer used

\maketitle

\begin{abstract}
   Unlike most engineered systems that facilitate the operation of nation-states, financial systems which we use for resource-allocation and the re-distribution of economic power regularly experience failure like no other established technology. In this setting, some may present decentralised finance(DeFi) as a cure-all, as it promises to reduce the influence of command-and-control economies and decentralise economic power. However, in this article I outline unique sociocultural vulnerabilities which may be amplified without careful regulation. If the history of human civilisation may teach us anything it is that we should tread carefully around romantic notions that develop around new technologies as the first version of a technological system is poorly regulated and often finds criminal use.
\end{abstract}

\section{The predictability of financial markets}

In general our expectations of engineered systems are that these are reliable to the degree that our predictions of their future behaviour are in agreement with their actual behaviour and so civilised societies should expect greater stability than hunter-gatherer societies. However, predicting the behaviour of financial systems is qualitatively different from predicting the behaviour of other systems as an economic agent, i.e. a net producer or net consumer, is both a participant and an observer. In particular, as the behaviour of the stock market is being constantly manipulated by the sum of agent interactions any plausible prediction of the future behaviour of the system is likely to modify the system’s actual behaviour. 

Yet, the situation is not completely hopeless. Given that financial markets depend upon human behaviour and this is not completely unpredictable…a deep understanding(if not a science) of human behaviour may help us regulate financial markets.

\section{Money as a mediator of metaphysical desire}

In general, humans purchase what they desire. All humans desire to satisfy their needs but as a general rule not all human desires satisfy biological needs. Money is rather unique as it is a mediator of both physical needs which are bounded and metaphysical needs which are unbounded.

In an important sense, monetary systems effectively establish a new creed whose numeracy belies an inextricable irrationality. This should come as no surprise as money has both utilitarian and metaphysical origins that mirror the rise of the nation-state.

This brings us one step closer to understanding why financial systems are particularly vulnerable to the madness of crowds.

\newpage

\section{Social networks, memes and mimetic behaviour}

A keen observer may note that humans are vulnerable to madness on both the supply and demand side of financial markets. On the demand side, bubbles would be improbable if not for the human weakness for mimetic behaviour on social networks(digital or otherwise). When people don’t know what to do or how to value something(ex. art), they tend to imitate each other. Today, thanks to the internet, memes may spread like wildfire.

Moreover, since most memes(and ideas) are unconstrained by rigorous scientific or empirical evaluations, most intellectual constructs are merely diversions. Their proliferation would not be possible if many educated humans were not willing participants in this irrational process whose large-scale consequences may be observed in the manner people vote with their feet. On the supply side, a large number of Darwin awards must be handed to the technically able financial engineers that develop new bubbles and therefore create the perfect conditions for important market crashes.

\section{Engineering bubbles on Wall Street, or the survival of the maddest}

If Wall Street is driven by mathematical reasoning, how can it systematically drive the world economy to the brink of collapse more than once and keep it in this critical state? The logic of Wall Street is to exploit the participant-observer weakness of financial markets and take it to the next level so we enter a player-regulator paradigm. In this paradigm they are both the owners of a casino and the main players. This is possible by using their financial clout to buy politicians.

In principle, financiers may rely upon portfolio theory which may be considered a rational approach to gambling. Given a finite number of assets, a financier will construct a portfolio by taking a weighted average of these assets:

\begin{equation}
\sum_{w_i \geq 0} w_i = 1 
\end{equation}

\begin{equation}
V = \sum_i w_i \cdot r_i	
\end{equation}

where $r_i$ is the expected monthly return on the ith investment.

Now, given that Wall Street financiers are gambling with the value of assets that are not their own(ex. housing securities) and they want the highest-possible returns there is a selection process at the psychological level for the highest-level of societal risk tolerance that brings Wall Street collectively just short of a catastrophic market crash. It follows that if you have a low tolerance for risk and reasonable levels of empathy you are unlikely to receive a promotion at JP Morgan or Goldman Sachs.

\section{The Wall Street moral code, or the lack thereof}

In principle, Wall Street institutions such as hedge funds may be a net positive for a nation-state and aren’t merely rent-seeking instruments. However, any honest person that investigates the manner Wall Street created the conditions for the mortgage crisis would realise that there is not merely a lack of regulatory oversight. Wall Street incarnates a culture and set of values that is everything except moral.

One might think that after the 2008 housing crisis, Wall Street financiers may have learned something. In fact, they are hard at work on the next crisis by profiting from student debt via student-loan-asset-backed-securities(SLABS). We are talking here about gambling not only on where people live but also gambling with the future incomes of American citizens.

In Wall Street there is a complete disregard for the damaging effects that this combination of crises, student debt and housing debt, have had on the American national psyche.

\newpage 

\section{Discussion}
If the reader may retain anything from this sociocultural analysis it is that any financial system, decentralised or not, must have safeguards in place for human vulnerabilities on both the supply and demand side of financial markets. This must include a well-respected moral code as well as enforceable rules and regulations.

I should add that the growth and momentum of the DeFi movement is symptomatic of the large-scale social dislocation felt across the board due to the unmitigated network effects that emerged from poorly regulated globalisation as well as the latent effects of the 2008 economic crisis. This tendency is effectively a bet against the Bretton Woods system that emerged after World War II, and so the IMF, Federal Reserve and World Bank would do well to heed these warning signs. I would like to make it equally clear that we can’t do away with command-and-control economies as they are crucial for establishing a rules-based global economic order. Furthermore, the Bretton Woods institutions which define this order remain absolutely essential for monitoring criminal/terrorist activity as well as analysing socioeconomic factors leading to state fragility in order to formulate economic development programs that mitigate against these risks.

In the near future, decentralised economies and centralised systems will have to cooperate within such a rules-based system which will probably emerge through a gradual process.

\section*{References}

\small

[1] Samantha Roy, CJ Ryan. The Next ‘Big Short’: COVID-19, Student Loan Discharge in Bankruptcy, and the SLABS Market. SSRN. 2020.

[2] Eli J. Campbell. Wall Street has been gambling with student loan debt for decades. Open Democracy. 2019.

[3] Daniel Andrei, Julien Cujean. Information percolation, momentum and reversal. Journal of Financial Economics. 2015.

[4] Percolation Theory Applied to Financial Markets: A Cluster Description of Herding Behavior Leading to Bubbles and Crashes. Maximilian G. A. Seyrich. 2015.

[5] Barbara Matthews. The Relevance of Bretton Woods in a Distributed, Cryptocurrency Age. The Bretton Woods Committee. 2019.  

[6] Pradumna B. Rana. From a Centralized to a Decentralized Global Economic Architecture: An Overview. Asian Development Bank Institute. 2013.

[7] World Development report: Conflict, Security, and Development. World Bank. 2011.

[8] Paul Allan Schott. Reference Guide to Anti-Money Laundering and Combating the Financing of Terrorism. 2006.

[9] Decentralised financial technologies. Financial Stability Board. 2019.

\end{document}